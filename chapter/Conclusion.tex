\section{Conclusion}
\label{sec: conclusion}

This thesis employed a machine-learning technique to reconstruct temperature data at weather stations, offering a computationally lightweight alternative to Numerical Weather Prediction (NWP).
The approach was specifically designed to utilize only the ERA5 reanalysis data as input, making it universally applicable to any weather station worldwide.
Trained models for each station, based on convolution layers, effectively identified local patterns in the ERA5 data, resulting in more accurate temperature predictions compared to the ERA5 data itself.
Upon validation against test data, the models consistently outperformed the ERA5 data, showcasing varying levels of accuracy across different datasets.
Despite limitations in training data availability, the models successfully corrected both major and minor biases present in the ERA5 data. However, the correlation coefficient wise performance was more closely tied to the quantity of training data.
Notably, the model trained for the weather station in Barbados displayed promising capabilities in reconstructing the diurnal cycle, a feature not adequately represented in the ERA5 data despite being trained on only two years of data.

Looking forward to future research, various extentions for improvement and application can be explored.
One potential area is the incorporation of additional sensors.
By integrating supplementary sensors, such as those for measuring solar radiation that the 3D printed weather stations are equipped with, the quality of reconstruction could be enriched.
This enhancement can be applied in two different stages.
Firstly, the additional data could be used in training, such that the model takes not only the ERA5 temperature as input but also the solar radiation in this example.
Secondly, there could be potential for a system that can benefit from other available sensor data during reconstruction as well. 

Towards the application of this approach, the thesis has developed an end-to-end software solution accompanied by a user-friendly web interface.
This interface enables the reconstruction of temperature data at any weather station globally without requiring extensive computer science knowledge.
It is crucial to note that the model's performance relies significantly on the quality and quantity of training data.
Hence, the software provides a platform for users to train and validate the model with their own data, accessible not only through the web interface but also automatically via the provided API.
Additionally, users can utilize the model's reconstruction results to assist in error-checking datasets by comparing them with actual measurements.
Subsequently, on the cleansed dataset, the model can be retrained, leading to improved accuracy.
The modularized design of the software allows for adjustments and extensions, such as incorporating different Reanalysis data sources.
If another reanalysis dataset is more accurate for a specific region, the software can be configured to utilize this dataset instead of ERA5, with the models subsequently retrained on the new data.
Furthermore, it can be explored how the model can be applied to existing weather forecast data, such as that from the European Centre for Medium-Range Weather Forecasts (ECMWF) or the National Oceanic and Atmospheric Administration (NOAA).
This could be a promising approach to improve the local forecast quality, as the model represents knowledge of the local weather patterns.