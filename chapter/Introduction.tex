\section{Introduction}
\label{sec:introduction}

% Weather stations are sparse in some regions

Weather station density varies greatly across the globe, depending on population density, economic development, and the availability of infrastructure. \cite{ortizbobea2021} While any weather station can experience downtime, the reliability of weather stations in regions with low station density is often low as well. So not only is downtime in regions where data is limited more likely but also more impactful because there are fewer neighboring stations to help compensate for the missing data.


A denser, more reliable network would benefit weather forecasting, helping to evacuate populations timely before natural disasters and from a global perspective aiding climate research. For example in East Africa, the weather station density is very low, but the region would be of great interest to the El Niño Southern Oscillation (ENSO) research. The irregular fluctuation between El Niño and La Niña phases affects the climate from the tropics to even higher-latitude regions through teleconnections. \cite{marchant2007, muita2021}  An innovative approach to increase the density of weather stations could be to use low-cost weather stations that could be 3D-printed and assembled by the local population \cite{muita2021}. Either way, low-cost weather stations have reliability issues and are prone to downtime which is outlined in section \ref{sec:3d_printed_stations}.

% Let's connect aerial data with local measurements

In light of the challenges posed by sparse weather station coverage, novel methodologies are required to address the reconstruction of missing weather data. One promising avenue involves the application of machine learning techniques, which offer a departure from traditional numerical reconstruction methods such as kriging, which are reliant on neighboring station data and are often computationally intensive \cite{chung2019kriging}. The application of machine learning in this case would be to connect numerical reanalysis data that describes the weather in grid cells meaning it's to some degree blurry, with the local patterns that lead to measurements at a weather station. This would allow for independent operation and would beat numerical methods in terms of needed computational resources needed by orders of magnitude \cite{kurth2023MLperformance,bi2023MLperformance,lam2023MLperformance} as the application of the trained machine-learning model. By leveraging available local data, these techniques, such as Convolutional Neural Networks (CNNs), can be trained to estimate weather conditions at a designated time by assimilating global numerical weather model data. Despite the inherent blurriness of aerial data provided in grid cells, these models are anticipated to discern and adapt to local weather patterns such that they become capable of transferring knowledge from the meta situation to the local situation. This paper aims to achieve reconstruction using that approach which will be further explained in \autoref{sec:design}.

% ERA5 0.25 hourly everywhere

The reanalysis of choice in this endeavor is the ECMWF Reanalysis v5 (ERA5), which covers the globe in grid cells of 0.25° x 0.25°. The data is available in hourly timesteps from 1940 to the present and contains a wide range of variables, such as temperature, precipitation, wind speed, and many more. \cite{era5}

% Let's go with the temperature

To prove the concept it's likely easiest to start with temperature data, meaning the 2m temperature variable from the ERA5 reanalysis will be used as input to the neural net.
%, one hour at a time, and the expected output will be the temperature at the weather station, during the same hour.