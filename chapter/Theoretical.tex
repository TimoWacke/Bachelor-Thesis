\section{Theoretical Background}
\label{sec:theory}

\subsection{Convolutional Neural Networks}
\label{subsec:cnn}

The application of CNNs in climate science has yielded several notable contributions. According to the sources provided, CNNs have excelled at extracting patterns and features from spatial data such as satellite imagery, radar data, and climate model outputs. This capability has enabled researchers to better understand and predict complex atmospheric and oceanic phenomena. For instance, CNNs have been employed to detect and localize extreme weather events from satellite data, enhancing early warning systems and disaster response efforts. Furthermore, CNNs have improved the ability to identify forced climate patterns and disentangle natural variability from human-induced climate change signals, advancing our understanding of climate dynamics and informing mitigation strategies.


Convolutional neural networks (CNNs) are a type of deep learning architecture inspired by the visual cortex of animals. They are designed to efficiently capture spatial and temporal dependencies in data through the use of learnable filters and hierarchical feature representations. Through the use of convolutional layers instead of fully connected layers, the architecture is able to preserve the spatial structure of the input data, making it particularly well-suited for image and video data. This approach not only simplifies pattern detection but also implies a reduction in the number of parameters, which minimizes the necessary computational resources.

Application of CNNs in climate science has yielded several notable contributions, including the reconstruction of the El Nino event of 1877 by Kadow et al. despite extremely limited data availability. \cite{kadow2020} 
\subsubsection*{Convolutional Layer}



\subsubsection*{Pooling Layer}

\subsection{Upscaling Layer}

\subsubsection*{Skip Connections}

\subsubsection*{Activation Function}

\subsubsection*{Supervised Learning}

\subsection{Reanalysis - ERA5}

\subsection{Weather Station Data Quality}