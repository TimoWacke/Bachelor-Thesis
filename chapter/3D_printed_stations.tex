\section{3D printed Weather Stations}
\label{sec:3d_printed_stations}

3D-Printed automated weather stations (3D-PAWS) are low-cost, compact automatic weather stations designed to expand meteorological observation networks, especially in data-sparse regions. They are equipped with sensors to measure various meteorological parameters like temperature, humidity, wind speed/direction, rainfall, and surface pressure. The station components (housing, sensor mounts, etc.) are 3D printed using durable plastic materials, making them lightweight and easy to deploy \cite{mwangi2017paws}.

% Comparison with Traditional Stations

The study compared data from a 3D-PAWS unit co-located with a manual synoptic weather station and a TAHMO automatic station in Kenya. Strong correlations were found between 3D-PAWS and the manual station for maximum temperature (r=0.59) and surface pressure (r=0.65). However, 3D-PAWS showed a high positive bias (~10°C) for minimum temperature readings, likely due to battery issues during non-daylight hours. Correlations were weaker for wind speed, wind direction, and rainfall between 3D-PAWS and the other stations \cite{mwangi2017paws}.

% Advantages and Challenges

3D-PAWS offer a cost-effective way to densify weather observation networks, especially in regions with limited resources. Their compact size and 3D-printed construction make them easy to transport and install in remote locations. Continuous monitoring and maintenance are crucial to ensure consistent data quality, as issues like battery failure can introduce biases. Further calibration and validation studies are needed to establish the reliability of 3D-PAWS data for various meteorological applications \cite{mwangi2017paws}.

In summary, 3D-PAWS represent an innovative approach to expanding weather monitoring capabilities using additive manufacturing technology. While showing promise, ongoing efforts are required to improve their data quality and reliability for widespread adoption \cite{mwangi2017paws}.
