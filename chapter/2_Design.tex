\section{Conceptual Framework and Methodology}
\label{sec:design}

\subsection{Concept}

\begin{figure}
    
\end{figure}

\subsection{Data Acquisition and Preprocessing}
The first step of the proposed methodology involves acquiring a dataset from a weather station. This dataset serves as the foundation for training and testing the Convolutional Neural Network (CNN). Upon obtaining the dataset, it is determined where temperature data is missing. While the weather station dataset is minute-based, data could be missing only for a few minutes within an hour instead of the full hour. To simplify the problem, the mean of the temperature values for each hour is calculated. And if all temperature values are missing for an hour, the hour is marked as missing. As explained in the introduction, the ERA5 reanalysis is hourly and covers the globe. The ERA5 data for the grid cells surrounding the weather station, at all the timesteps since the weather station data starts is obtained.

The ERA5 data then needs to be cropped to the neighboring grid cells, while centering the cutout as close to the weather station as possible. In the next step, the ERA5 data is divided into two datasets: one with all the hours marked as missing and one with all the hours marked as present. Until the model is trained, only the dataset with all the hours marked as present will be used.

\subsection{Model Setup and Training}
To determine if and to which extent the model learned to reconstruct the missing data, after we trained it the weather station dataset and the corresponding ERA5 dataset are split again into a pair of station and ERA5 data for training and one for validation. The training set is used to train the model, while the validation set is reserved to evaluate the model's performance. With the datasets prepared, the next phase involves configuring and training the Convolutional Neural Network (CNN) for the temperature reconstruction task. The CNN architecture is tailored to accept input in the form of 8x8 grid cells centered around the weather station's location. Employing a supervised learning approach, the CNN is trained using pairs of hourly temperature data from the weather station and corresponding grid cell data from ERA5. The training process iteratively feeds batches of data into the CNN, fine-tuning its parameters to minimize prediction errors and optimize accuracy in reconstructing missing temperature values.

\subsection{Model Evaluation}
Following the training phase, the CNN's performance is evaluated using the validation set. The model's capacity to accurately reconstruct missing temperature data at the weather station is scrutinized against ground truth values. This evaluation step serves to gauge the CNN's proficiency in capturing intricate weather patterns and producing precise temperature estimations. For that, the root mean squared error (RMSE) and the correlation coefficient are calculated. The RMSE is a measure of the differences between predicted and observed values, while the correlation coefficient quantifies the strength and direction of the linear relationship between the two datasets.

\subsection{Application to New Data}
Upon successful training and validation, the CNN is applied to the ERA5 dataset for the hours where station data is missing. These surrounding grid cell data from ERA5 are then fed into the trained CNN, which generates predictions for the missing temperature values at the weather station. Following the same approach as for validation the CNN reconstructs the weather data from the input grid cell data, as the model with its trained parameters is applied hour by hour. Thus the result is not a single continuous time series but a series of hourly predictions.