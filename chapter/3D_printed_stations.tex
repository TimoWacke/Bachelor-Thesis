\section{3D Printed Weather Stations}
\label{sec:3d_printed_stations}

3D-Printed Automated Weather Stations (3D-PAWS) represent an innovative approach to enhancing meteorological observation networks, particularly in regions with limited data availability. Utilizing 3D printing technology, these stations are constructed from lightweight and durable plastic, facilitating easy deployment and maintenance. Equipped with sensors to measure parameters such as temperature, wind, solar radiation, and precipitation, 3D-PAWS ensure comprehensive weather data collection \cite{mwangi2017paws}.

At the heart of the 3D-PAWS system lies a Raspberry Pi, responsible for sensor control and data logging. Powered by solar panels and batteries, these stations can operate autonomously in remote areas. Data transmission is facilitated through either cell or satellite modems, enabling real-time updates accessible under open access policies. This approach not only ensures continuous data flow but also promotes transparency and accessibility in meteorological data sharing \cite{mwangi2017paws}.

Cost-effectiveness is a significant advantage of 3D-PAWS. Structural components for these stations are locally sourced, with an approximate cost of \$100. Additional expenses include the Raspberry Pi (\$60), micro-sensors (\$100), and power solutions (\$50). Communication costs remain minimal, with cell modems priced at \$30 and satellite modems at \$50 \cite{mwangi2017paws}.

Recent advancements in 3D printing have substantially reduced the production costs of these weather stations. With capable 3D printers now available for around \$500, the production and deployment of 3D-PAWS have become more accessible and affordable. This democratization of weather station technology holds the potential to revolutionize meteorological observation, especially in regions where traditional stations are impractical or prohibitively expensive.

In summary, 3D-PAWS offer a significant advancement in meteorological observation, combining affordability, ease of deployment, and comprehensive data collection capabilities. By leveraging 3D printing and modern sensor technology, these stations can address critical gaps in weather data networks worldwide, thereby improving weather prediction and climate research efforts \cite{muita2021}.

The primary objective of the 3D-PAWS initiative is to save lives through improved early warning systems for catastrophic events. The project is sponsored by the University Corporation for Atmospheric Research (UCAR) and the US National Weather Service International Activities Office (NWS IAO), with support from the USAID Office of U.S. Foreign Disaster Assistance (OFDA) \cite{3dpaws_manual}.

% \subsection{Evaluation of Sensor Performance}
3D-PAWS systems are being rigorously evaluated at various locations, including the NCAR Marshall Field Site in Boulder, CO, the NOAA Testbed facility in Sterling, VA, and selected international sites. These evaluations assess sensor performance in diverse climatic conditions, ensuring the reliability and accuracy of data collected by the stations.

% \subsection{Pilot Networks and Data Accessibility}
Pilot networks of 3D-PAWS have been deployed in the United States and over 10 other countries globally. Real-time data from these stations are accessible through the CHORDS project data servers, facilitating their utilization in hydrometeorological applications worldwide \cite{3dpaws_manual}.

% \subsection{Benefits and Applications}
The observations provided by 3D-PAWS have diverse applications, including regional weather forecasting, early alert systems for natural disasters, agricultural monitoring, and health surveillance. These applications underscore the potential impact of 3D-PAWS in mitigating risks associated with weather-related events and enhancing resilience in vulnerable communities. \cite{3dpaws_manual}.